\section{审查 BIM 平台以支持安全规划}
\subsection{已知问题}
对一些商业化 BIM 平台的支持安全规划的能力进行了检查。若干功
能性先决条件被认为对于启用基于 BIM 的安全规划非常重要。它们的名
单如下:

\begin{enumerate}
    \item 调度和模拟:建筑业的复杂和动态特性及其现场工作模式已经被广泛的认同。
    为了在施工过程中发现和预防安全隐患,需要将工程进度与
    BIM 联系起来。此外,如何根据施工进度计划进行施工进度的可
    视化,提高施工人员的安全意识和沟通能力是施工进度计划可视
    化的关键。    
    \item 建模:建筑安全不仅是管理或控制工人的安全行为,还包括设计、采
    购、安装和拆除安全和节奏设备,如护栏、脚手架和安全网或挂
    钩。为了可视化和定量化的目的,在 BIM 中对这些临时对象进行
    设计和建模是非常必要的。因此,一个理想的平台需要能够创建和修改模型项,并提供可视化。
    \item 施工现场布置建模与可视化:认识施工现场物流的重要性和施工现场
    的动态性,从安全角度考虑施工现场布置的重要性。现场布局的
    建模和可视化能力可以支持详细、准确的现场逻辑分析,从而提
    高生产率和工作场所安全性。
    \item 模型格式:如前所述,使用 IFC 数据格式允许对各种 BIM 创作工具
    中包含的模型进行更一般的检查。
    \item 规则检查能力:BIM 平台配备了自己的规则引擎,可以为用户提供自
    定义或用户配置的规则检查过程安全规则的机会。
\end{enumerate}


几个现有的商业化 BIM 软件解决方案的比较及其结合安全的潜力见表 \ref{tb:CP}

\begin{table}[thbp]
    \caption{四种 BIM 应用的比较分析}
    \begin{center}
        \begin{tabular}{@{}lcccccc@{}}
            \toprule
            \multicolumn{1}{c}{\multirow{2}{*}{\textbf{应用软件}}} & \multicolumn{6}{c}{\textbf{功能}} \\ \cmidrule(l){2-7} 
            \multicolumn{1}{c}{} & \textbf{计划} & \textbf{模拟} & \textbf{主体建模} & \textbf{施工平面图建模} & \textbf{基于 IFC 格式} & \textbf{规则检查} \\ \midrule
            \textbf{Autodesk Revit} & - & - & √ & √ & - & - \\
            \textbf{Autodesk Navisworks} & √ & √ & - & - & - & - \\
            \textbf{Solibri Model Checker} & - & - & - & - & √ & √ \\
            \textbf{Tekla Structures} & √ & √ & √ & - & - & - \\ \bottomrule
            \end{tabular}
    \end{center}
    \label{tb:CP}
\end{table}
\newpage
SMC 作为一种基于 BIM 的工具的优势在于它能够使用 IFC 数据交换
格式,这使得基于 BIM 的建模软件的检查工作可以成为一种独立的工具来使用。规则检
查功能和用户界面还提供了合并安全解决方案的可能。然而,尽管自
动化被用来进行日常的检查工作,仍然需要人工对所有与安全相关的临
时设备和结构进行建模,这些设备和结构在基于 BIM 的建模软件中现有
的对象库中有的不被支持,有的缺乏项目。根据项目的规模或复杂程度,冗长的模型修
改过程通常需要几天甚至几周的时间。在 Navisworks 上也发现了类似的
问题,由于缺乏建模功能,很难添加与安全相关的设备。施工现场的动
态特性无论是在施工现场管理系统中还是在施工现场管理系统中都无法
体现出来,这使得施工现场管理系统在不同的施工阶段都需要进行规范
检查。由于本文侧重于建筑物相关的跌落危险,场地布局建模和可视化
不属于本文研究的范围。因此,在比较分析的基础上,选择 Tekla Structures
作为本文研究的安全规则检测算法的实现平台。另外,为了使建筑信息
模型能够应用于施工过程规划或分析,需要进行大量的现场布局和操作
建模工作。