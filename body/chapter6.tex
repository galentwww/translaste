\section{总结}
开发的建筑物信息模型跌落危险检测与预防安全规则检查平台已在
两个案例中成功实现。该算法能够检测混凝土板和前缘中潜在的坠落危
险的位置,并提供相应的坠落防护装置的安装指南(例如,材料清单、可
视化),实际上解决了 BIM 中确定的坠落危险。结果表明,该方法在安
全设计和规划阶段能够有效地探测和可视化潜在的跌落危险。

由于自动生成的防坠落计划必须由安全专家检查,但如果遵循其他安全指南或有更好地实践方法,
也允许进行调整。开发的平台显示出强大的潜力,可以创建基于 BIM 的安全计划,
可视化施工进度表中的安全性,安排安装和拆除工作,
并提供包括永久性建筑部件和临时安全设备在内的选项和程序,并模拟它们。

未来的一个目标可能导致安全行业最佳做法的重大变化,即基于 
BIM 的安全规划可能成为标准建筑物建设规划过程的一部分。基于 BIM 
的建模还可以增加安全理解和交流,尤其是在工程设计和施工计划阶段。
由于建筑信息模型中的施工进度表与模型对象之间只有几个星期的时间,
因此在施工对象的结构模型、施工进度表或原始安装顺序发生变化之后,
有必要检查模型是否发生了变化,并更新检查结果。开发的系统通过消
除设计和规划阶段的危险,确保采购安全设备,并在需要时在正确的地
点和时间准备好安装,协助人类决策制定者进行这一审查过程。

在考虑基于 BIM 的安全规划自动化取代人工建模的优点时,发现通
过减少时间和人工建模工作,自动化有可能显著提高基于 BIM 的计划编
制过程。一旦发生设计变更,就需要使用手动建模进行额外的工作。需
要时间和其他人力资源仔细检查模型,以确保模型护栏或其他防护设备
仍然有效和准确。虽然人工建模的优点是人类参与了每一步,但是它很
耗时,而且可能容易出错。一个系统,提供自动化和一致的结果,然后
由一个人审查,可以提供更频繁和更快的更新。

目前对所开发系统应用的关注包括:(1)以模型为基础的设计和施工图
纸在动态的施工环境中经常发生变化,因此有必要定期检查模型,以确保
安全状况识别和纠正措施是最新的;(2)自动安全模型的质量和详细程度
可能需要进一步发展,以满足设计部门和建筑工地从业员的实际需要,
并非所有模型都能提供安全规划及检查所需的资料,例如墙体与平板部
分之间的关系(例如连接)可能未能妥善建立或正确建模。这可能会防止
识别拆除护栏的建设进展。这种与安全相关的设计标准或者延伸的 BIM
要求需要被研究和正式化。