\section*{\zihao{-2} \centering 摘 ~~ 要}

\vskip0.5cm
建筑信息模型在建筑设计和建筑规划中的应用正在迅速发展。基于 BIM 的建模和 4D 仿真(3D 和调度)为安全和物流
应用带来了许多好处。然而,到目前为止,在建模和规划安全过程方面只开发了有限的自动化。本研究的目的是探
讨如何在建筑项目的规划阶段的早期识别和消除潜在的跌落危险,这些危险是不知不觉地纳入施工进度表的。

本文首
先介绍了国内外建筑安全与 BIM 的研究概况。然后,提出了一个包含 BIM 安全规则自动检查算法的框架。开发的原
型使用模型进行了测试,其中包括芬兰的一个办公室和一个住宅建筑项目。第一个案例研究突出了手动与自动交配
的安全建模跌倒保护系统的比较。它还描述了保护安全设备模型的多种设计和建成方案的细节。第二个案例研究展
示了将该框架应用于项目进度计划的结果。它特别模拟跌落危险的检测和预防。这项工作的贡献是一个自动化的规
则检查框架,有效地将安全融入 BIM,并为从业人员提供一种检测和预防跌落相关危险的方法。会上还讨论了有关
已开发的先进型产品商业化的公开问题,并探讨了扩大传统安全管理做法对解决实地安全问题可能产生的影响。


{\zihao{4} \heiti 关键词: } \zihao{-4}建筑信息模型,建筑安全规则和代码检查,防止坠落危险,通过设计进行计划、调度和模拟预防
\addcontentsline{toc}{section}{摘要}

\clearpage
\section*{\zihao{-2} \centering \textbf{Abstract} }
   %用了Times New Roman字体来美化观感

   The applications of Building Information Modeling (BIM) in building design and construction planning
   are growing rapidly. BIM-based modeling and 4D simulation (3D and schedule) has brought many
   benefits to safety and logistics applications as well. However, only limited automation in modeling
   and planning safety processes has been exploited so far. The objective of this study is to investigate
   how potential fall hazards that are unknowingly built into the construction schedule can be identified
   and eliminated early in the planning phase of a construction project. A survey of research on construction
   safety and BIM is presented first. Then, a framework was developed that includes automated safety rule-
   checking algorithms for BIM. The developed prototype was tested using models including an office and a
   residential building project in Finland. The first case study highlights the comparison of manual vs. auto-
   mated safety modeling of fall protective systems. It also describes the details to multiple design and as-
   built scenarios where protective safety equipment is modeled. The second case study presents results of
   applying the framework to the project schedule. It specifically simulates fall hazard detection and pre-
   vention. The contribution of this work is an automated rule-checking framework that integrates safety
   into BIM effectively and provides practitioners with a method for detecting and preventing fall-related
   hazards. Presented are also discussions of open issues regarding commercialization of the developed pro-
   totype and considerations which explore what impact it might have on resolving safety issues in the field
   by extending traditional safety management practices.

\textbf{\zihao{4} Key Words:} Building Information Modeling,Construction safety rule and code checking,Fall hazard prevention,Planning;scheduling;and simulation,Prevention through Design
\addcontentsline{toc}{section}{Abstract}




